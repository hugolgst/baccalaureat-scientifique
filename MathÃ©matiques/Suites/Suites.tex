\documentclass[a4paper]{article}
\usepackage[utf8]{inputenc}
\usepackage[a4paper, left=2.5cm, right=2.5cm, top=2cm, bottom=2cm]{geometry}
\usepackage{lmodern}
\usepackage[T1]{fontenc}
\usepackage{graphicx}
\usepackage{amssymb}
\usepackage[utf8]{inputenc}
\usepackage{csquotes}

\renewcommand{\thesection}{\Alph{section}}
\renewcommand{\thesubsection}{\arabic{subsection}}
\renewcommand{\thesubsubsection}{(\emph{\alph{subsubsection}})}

\title{Suites}
\author{Hugo Lageneste}
\date{Janvier 2020}

\begin{document}

{Mathématiques - Suites}

\begin{center}
 \newcommand{\HRule}{\rule{\linewidth}{0.5mm}}
 {\huge \bfseries Suites}\\[0.1cm]
\end{center}

\section{Étude}
\subsection{Majorée, minorée, bornée}

{Dire que \(\left(u_n\right)\) est majorée (resp. minorée) par un réel \(M\) (resp. \(m\)) c'est:}

\begin{center}
	\(u_n \leqslant M\) (resp. \(m \leqslant u_n\))
\end{center}
{Une bornée est majorée et minorée:}

\[m \leqslant u_n \leqslant M\]

\subsection{Arithmétique}

{Soit $\left(u_n\right)$ une suite arithmétique définie par $u_n = u_0 + nr$.}\\
{La somme des termes est}

\[S=\frac{(n+1)(u_0+u_n)}{2}\]

\subsection{Géométrique}

{Soit $\left(u_n\right)$ une suite géométrique définie par $u_n = u_0 \times q^n$.}\\
{La somme des termes est}

\[S=u_0 \times \frac{1-q^n}{1-q}\]

\section{Limites}
\subsection{Étude}

{Une limite de suite ne peut être étudiée qu'en $+\infty$.}\\

{Une suite convergente admet une limite finie. Toute suite convergente est bornée.}\\
{Ainsi, une suite est divergente si et seulement si elle n'est pas convergente, ou que sa limite est indéterminée.}

\subsection{Suite géométrique}

Soit un réel $q$:
\begin{center}
	{Si $-1 < q < 1$ alors $\lim\limits_{n \rightarrow +\infty} q^n=0$}\\
	{Si $1 < q$ alors $\lim\limits_{n \rightarrow +\infty} q^n=+\infty$}\\
	{Si $q \leqslant -1$ alors $\lim\limits_{n \rightarrow +\infty} q^n$ indéterminée}\\
	{Si $q=1$ alors $\lim\limits_{n \rightarrow +\infty} q^n=1$}
\end{center}

\subsection{Calcul de limites}

{Soient $\left(u_n\right)$ et $\left(v_n\right)$}\\

\subsubsection{Somme}

\begin{center}
	\begin{tabular}{|c|c|c|c|c|c|c|}
  		\hline
  		$\lim\limits_{n \rightarrow +\infty} u_n$ & $L$ & $L$ & $L$ & $-\infty$ & $+\infty$ & $+\infty$ \\
  		\hline
  		$\lim\limits_{n \rightarrow +\infty} v_n$ & $L\prime$ & $+\infty$ & $-\infty$ & $-\infty$ & $+\infty$ & $-\infty$ \\
  		\hline
  		$\lim\limits_{n \rightarrow +\infty} u_n + v_n$ & $L + L\prime$ & $+\infty$ & $-\infty$ & $-\infty$ & $+\infty$ & Indéterminée \\
  		\hline
	\end{tabular}
\end{center}

\subsubsection{Produit}

\begin{center}
	\begin{tabular}{|c|c|c|c|c|c|c|c|c|c|}
  		\hline
  		$\lim\limits_{n \rightarrow +\infty} u_n$ & $L$ & $L > 0$ & $L > 0$ & $L < 0$ & $L < 0$ & $+\infty$ & $+\infty$ & $-\infty$ & $0$ \\
  		\hline
  		$\lim\limits_{n \rightarrow +\infty} v_n$ & $L\prime$ & $+\infty$ & $-\infty$ & $+\infty$ & $-\infty$ & $+\infty$ & $-\infty$ & $-\infty$ & $-\infty$ ou $-\infty$ \\
  		\hline
  		$\lim\limits_{n \rightarrow +\infty} u_n \times v_n$ & $L \times L\prime$ & $+\infty$ & $-\infty$ & $-\infty$ & $+\infty$ & $+\infty$ & $-\infty$ & $+\infty$ & Indéterminée \\
  		\hline
	\end{tabular}
\end{center}

\subsubsection{Quotient si $\lim\limits_{n \rightarrow +\infty} v_n \neq 0$}

\begin{center}
	\begin{tabular}{|c|c|c|c|c|c|c|c|}
  		\hline
  		$\lim\limits_{n \rightarrow +\infty} u_n$ & $L$ & $L$ & $+\infty$ & $+\infty$ & $-\infty$ & $-\infty$ & $-\infty$ ou $-\infty$ \\
  		\hline
  		$\lim\limits_{n \rightarrow +\infty} v_n$ & $L \neq 0$ & $-\infty$ ou $-\infty$ & $L\prime > 0$ & $L\prime < 0$ & $L\prime > 0$ & $L\prime < 0$ & $-\infty$ ou $-\infty$\\
  		\hline
  		$\lim\limits_{n \rightarrow +\infty} \frac{u_n}{v_n}$ & $\frac{L}{L\prime}$ & $0$ & $+\infty$ & $-\infty$ & $-\infty$ & $+\infty$ & Indéterminée \\
  		\hline
	\end{tabular}
\end{center}

\subsubsection{Quotient si $\lim\limits_{n \rightarrow +\infty} v_n = 0$}

\begin{center}
	\begin{tabular}{|c|c|c|c|c|c|}
  		\hline
  		$\lim\limits_{n \rightarrow +\infty} u_n$ & $L>0$ ou $+\infty$ & $L>0$ ou $+\infty$ & $L<0$ ou $+\infty$ & $L<0$ ou $+\infty$ & $0$ \\
  		\hline
  		$\lim\limits_{n \rightarrow +\infty} v_n$ & $0^+$ & $0^-$ & $0^+$ & $0^-$ & $0$ \\
  		\hline
  		$\lim\limits_{n \rightarrow +\infty} \frac{u_n}{v_n}$ & $+\infty$ & $-\infty$ & $-\infty$ & $+\infty$ & Indéterminée \\
  		\hline
	\end{tabular}
\end{center}

\section{Comparaison}
\subsection{Théorème des gendarmes/d'encadrement}

{Soient $\left(u_n\right)$, $\left(v_n\right)$ et $\left(w_n\right)$}\\
{Si} 
\begin{itemize}
	\item{$u_n \leqslant v_n \leqslant w_n$}
	\item{$\left(u_n\right)$ et $\left(w_n\right)$ convergent toutes deux vers $L$}
\end{itemize}
{Alors $\left(v_n\right)$ converge vers $L$}

\section{Raisonnement par récurrence}
\subsection{Initialisation}

{On vérifie si la propriété est vraie au premier rang $p$.}

\subsection{Hérédité}

{On vérifie si cette propriété est vraie pour un certain rang $k$.}

\subsection{Conclusion}

{La propriété est donc vraie pour tout entier naturel à partir de $p$.}

\end{document}












































\end{document}

%! \bookmark[document=Ym9va+wCAAAAAAQQMAAAANCToywiuwr7kdOmAUQ+YwNMTuDrqMHq34BjmRbAdtUM6AEAAAQAAAADAwAAAAgAKAUAAAABAQAAVXNlcnMAAAAEAAAAAQEAAGh1Z28JAAAAAQEAAERvY3VtZW50cwAAAA0AAAABAQAAQmFjY2FsYXVyw6lhdAAAAA8AAAABAQAATWF0aGXMgW1hdGlxdWVzABQAAAABBgAAEAAAACAAAAAsAAAAQAAAAFgAAAAIAAAABAMAAKmPEwAAAAAACAAAAAQDAACujxMAAAAAAAgAAAAEAwAA6H97AAAAAAAIAAAABAMAAAB4vAAAAAAACAAAAAQDAABQyLwAAAAAABQAAAABBgAAjAAAAJwAAACsAAAAvAAAAMwAAAAIAAAAAAQAAEHB7lXkwcFpGAAAAAECAAACAAAAAAAAAA8AAAAAAAAAAAAAAAAAAAAIAAAABAMAAAMAAAAAAAAABAAAAAMDAAD1AQAACAAAAAEJAABmaWxlOi8vLwwAAAABAQAATWFjaW50b3NoIEhECAAAAAQDAAAAwD9kdAAAAAgAAAAABAAAQcHI2A7qmnAkAAAAAQEAAEMxOTI0NzRGLTA3ODEtNDY2Ni1CNzRDLUZGQjIxQkU3QUQ3MhgAAAABAgAAgQAAAAEAAADvEwAAAQAAAAAAAAAAAAAAAQAAAAEBAAAvAAAAAAAAAAEFAADMAAAA/v///wEAAAAAAAAAEAAAAAQQAABwAAAAAAAAAAUQAADcAAAAAAAAABAQAAAIAQAAAAAAAEAQAAD4AAAAAAAAAAIgAADUAQAAAAAAAAUgAABEAQAAAAAAABAgAABUAQAAAAAAABEgAACIAQAAAAAAABIgAABoAQAAAAAAABMgAAB4AQAAAAAAACAgAAC0AQAAAAAAADAgAADgAQAAAAAAAAHAAAAoAQAAAAAAABHAAAAgAAAAAAAAABLAAAA4AQAAAAAAABDQAAAEAAAAAAAAAA==]
