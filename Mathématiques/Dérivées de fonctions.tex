\documentclass[a4paper]{article}
\usepackage[utf8]{inputenc}
\usepackage[a4paper, left=2.5cm, right=2.5cm, top=2cm, bottom=2cm]{geometry}
\usepackage{lmodern}
\usepackage[T1]{fontenc}
\usepackage{graphicx}
\usepackage{amssymb}
\usepackage[utf8]{inputenc}
\usepackage{pgfplots}
\pgfplotsset{width=6cm,compat=1.9}
\usepackage{multicol}
\usepackage{csquotes}

\renewcommand{\thesection}{\Alph{section}}
\renewcommand{\thesubsection}{\arabic{subsection}}
\renewcommand{\thesubsubsection}{(\emph{\alph{subsubsection}})}

\title{Dérivation}
\author{Hugo Lageneste}
\date{Février 2020}

\begin{document}

{Mathématiques - Dérivation}

\begin{center}
 \newcommand{\HRule}{\rule{\linewidth}{0.5mm}}
 {\huge \bfseries Dérivation}\\[0.1cm]
\end{center}

\section{Nombre dérivé}
\subsection{Taux d'accroissement}

{Le taux d'accroissement est défini par}

\[\lim\limits_{h \rightarrow 0}\frac{f(a+h)-f(a)}{h}=f\prime (a)\]

\subsection{Tangente}

{L'équation de la tangente à une courbe représentative de fonction en un point a est définie par}

\[y=f\prime(a)(x-a)+f(a)\]

\section{Fonction dérivée}

\subsection{Dérivées des fonctions usuelles}
\begin{center}
	\begin{tabular}{|c|c|c|c|}
  		\hline
  		$f(x)$ & $f\prime(x)$ & $D_f$ & $D_{f\prime}$ \\	
  		\hline
  		$\lambda$ & $0$ & $\mathbb{R}$ & $\mathbb{R}$ \\
  		\hline
  		$x$ & $1$ & $\mathbb{R}$ & $\mathbb{R}$ \\
  		\hline
  		$x^n$ $(n \geq 1)$ & $nx^{n-1}$ & $\mathbb{R}$ & $\mathbb{R}$ \\
  		\hline
  		$\frac{1}{x^n}$ $(n \geq 1)$ & $-\frac{n}{x^{n+1}}$ & $\mathbb{R}^*$ & $\mathbb{R}^*$ \\
  		\hline
  		$\sqrt{x}$ & $\frac{1}{2\sqrt{x}}$ & $\mathbb{R}^+$ & $\mathbb{R}^+*$ \\
  		\hline
	\end{tabular}
\end{center}

\subsection{Opérations sur les dérivées}
\begin{center}
	\begin{tabular}{|c|c|}
  		\hline
  		$f$ & $f\prime$ \\
  		\hline
  		$\lambda u$ & $\lambda u\prime$ \\
  		\hline
  		$u + v$ & $u\prime + v\prime$ \\
  		\hline
  		$uv$ & $u\prime v + uv\prime$ \\
  		\hline
  		$\frac{1}{v}$ & $-\frac{v\prime}{v^2}$ \\
  		\hline
  		$\frac{u}{v}$ & $\frac{u\prime v - uv\prime}{v^2}$ \\
  		\hline
  	\end{tabular}
\end{center}

\subsection{Dérivées de fonctions composées}
\begin{center}
	\begin{tabular}{|c|c|}
  		\hline
  		$f$ & $f\prime$ \\
  		\hline
  		$u^n$ $(n \geq 1)$ & $nu\prime u^{n-1}$ \\
  		\hline
  		$\sqrt{u}$ (si $u(x) > 0$) & $\frac{u\prime}{2\sqrt{u}}$ \\
  		\hline
  	\end{tabular}
\end{center}

\end{document}⏎